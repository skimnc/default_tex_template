%===================================================================%
%===================================================================%
% 0. Preamble
%      - Page layout
%      - Custom commands
%===================================================================%
%===================================================================%
% !TEX root = root_document.tex
%----------------------------------------------------------------------------------------------------------------------%

\newif\ifnever\neverfalse


%===================================================================%
% Theorem + section customization
%===================================================================%
\newcommand*{\CopyCounter}[2]{%
  \expandafter\def\csname c@#2\endcsname{\csname c@#1\endcsname}%
  \expandafter\def\csname p@#2\endcsname{\csname p@#1\endcsname}%
  \expandafter\def\csname the#2\endcsname{\csname the#1\endcsname}}
\newcounter{Theorem}
\numberwithin{Theorem}{section}
\CopyCounter{Theorem}{Proposition}
\CopyCounter{Theorem}{ProposedProblem}
\CopyCounter{Theorem}{Property}
\CopyCounter{Theorem}{Claim}
\CopyCounter{Theorem}{Lemma}
\CopyCounter{Theorem}{Corollary}
\CopyCounter{Theorem}{Conjecture}
\CopyCounter{Theorem}{Definition}
\CopyCounter{Theorem}{Example}
\CopyCounter{Theorem}{Remark}
\CopyCounter{Theorem}{Question}
\CopyCounter{Theorem}{Condition}
\CopyCounter{Theorem}{Criterion}
\CopyCounter{Theorem}{Observation}
\theoremstyle{plain}%% needs amsthm.sty
\newtheorem{thm}[Theorem]{Theorem}
\providecommand\Theoremname{Theorem}
\newtheorem{problem}[ProposedProblem]{Problem} % RT changed Proposed Problem -> Problem
\providecommand\ProposedProblemname{ProposedProblem}
\newtheorem{prop}[Proposition]{Proposition}
\providecommand\Propositionname{Proposition}
\newtheorem{property}[Property]{Property}
\providecommand\Propertyname{Property}
\newtheorem{claim}[Claim]{Claim}
\providecommand\Claimname{Claim}
\newtheorem{lemma}[Lemma]{Lemma}
\providecommand\Lemmaname{Lemma}
\newtheorem{cor}[Corollary]{Corollary}
\providecommand\Corollaryname{Corollary}
\theoremstyle{definition}%% needs amsthm.sty
\newtheorem{definition}[Definition]{Definition}
\newtheorem{example}[Example]{Example}
\newtheorem{remark}[Remark]{Remark}
\providecommand\Remarkname{Remark}
\newtheorem{conjecture}[Conjecture]{Conjecture}
\providecommand\Conjecturename{Conjecture}
\newtheorem{quest}[Question]{Question}
\newtheorem{cond}[Condition]{Condition}
\newtheorem{crit}[Criterion]{Criterion}
\newtheorem{observ}[Observation]{Observation}

\newif\ifnever\neverfalse


%===================================================================%
%===================================================================%
%\def\chaptername{Chapter}
%\def\sectionname{Section}
%\def\subsectionname{Subsection}
%\def\subsubsectionname{Subsubsection}
%\def\paragraphname{Paragraph}
%\renewcommand{\baselinestretch}{2}


%===================================================================%
%===================================================================%
%===================================================================%
%======================= USEFUL COMMANDS ==========================%
%===================================================================%
%===================================================================%
%===================================================================%

\newcommand{\bm}[1]{\mbox{\boldmath${#1}$}}

%----------------------------------------------------------------------------------------------------------------------%
% Domain related
%----------------------------------------------------------------------------------------------------------------------%
\newcommand{\domain}{\Omega}
%\newcommand{\cdomain}{\bar{\Omega}}
\newcommand{\cdomain}{\widebar{\domain}}
\newcommand{\boundary}{\partial \domain}
\newcommand{\inside}{\domain \backslash \partial \domain}

%----------------------------------------------------------------------------------------------------------------------%
% Bold, hats, etc...
%----------------------------------------------------------------------------------------------------------------------%
\newcommand{\bx}[1]{\mbox{\boldmath$x_{#1}$}}
\newcommand{\by}[1]{\mbox{\boldmath$y_{#1}$}}

\newcommand{\xBar}{\bm{\bar{x}}}
\newcommand{\xtilde}{\tilde{x}}
\newcommand{\xhat}{\bm{\hat{x}}}
\newcommand{\yhat}{\bm{\hat{y}}}

\newcommand{\bxr}{\bx{r}}
\newcommand{\bxn}{\bx{n}}

\newcommand{\x}{\bm{x}}
\newcommand{\p}{\bm{p}}
\newcommand{\bs}{\bm{s}}
\newcommand{\bt}{\bm{t}}

%----------------------------------------------------------------------------------------------------------------------%
% Potentially unused
%----------------------------------------------------------------------------------------------------------------------%
\newcommand{\widebar}{\overline}
\DeclareMathOperator*{\argmax}{argmax}
\DeclareMathOperator*{\argmin}{argmin}
\DeclareMathOperator{\Forall}{\;\forall\,}

\newcommand{\maxb}{B}
\newcommand{\bud}{b}
\newcommand{\budp}{\beta}
\newcommand{\bq}{\begin{equation}}
\newcommand{\eq}{\end{equation}}
\newcommand{\Valn}{U}

%----------------------------------------------------------------------------------------------------------------------%
% Mathbb
%----------------------------------------------------------------------------------------------------------------------%
\newcommand{\R}{\mathbb{R}}					% real numbers

%----------------------------------------------------------------------------------------------------------------------%
% Mathcal
%----------------------------------------------------------------------------------------------------------------------%
\newcommand{\Neigh}{\mathcal{N}}				% neighbors(x)
\newcommand{\Bd}{\mathcal{B}}				% budget levels
\newcommand{\TT}{\mathcal{T}}				% operator
\newcommand{\PF}{\mathcal{PF}}				% Pareto Front
\newcommand{\Path}{\mathcal{P}}				% Path

%----------------------------------------------------------------------------------------------------------------------%
% Useful containers
%----------------------------------------------------------------------------------------------------------------------%
\newcommand{\VEC}[1]{\text{vec}\left({#1}\right)}	% vector
\newcommand{\abs}[1]{\left|{#1}\right|}			% absolute value	| |
\newcommand{\braces}[1]{\left[{#1}\right]}			% braces			[ ]
\newcommand{\cbraces}[1]{\left\{{#1}\right\}}		% curly Braces		{ }
\newcommand{\ip}[1]{\left\langle{#1}\right\rangle}	% inner product 	< >
\newcommand{\norm}[1]{\left\|{#1}\right\|}			% norm			|| ||
\newcommand{\parens}[1]{\left({#1}\right)}			% parentheses		( )
\newcommand{\floor}[1]{\left\lfloor{#1}\right\rfloor}	% floor
\newcommand{\ceil}[1]{\left\lceil{#1}\right\rceil}		% ceil
\newcommand{\mybox}[1]{\boxed{#1}}			% boxed



%===================================================================%
%===================================================================%
%===================================================================%
%===================== END USEFUL COMMANDS ========================%
%===================================================================%
%===================================================================%
%===================================================================%

%AV
%===================================================================%
% Margin fixes
%===================================================================%
%\usepackage{psfig}
\newcommand{\marginfix}{
\setlength{\parskip}{0.01cm}
\setlength{\textwidth}{6.0in}
%latex measures from 1 in down and over for some reason
\setlength{\oddsidemargin}{-0.0 in}
\setlength{\evensidemargin}{0.0 in}
\setlength{\topmargin}{-0.5in}
\setlength{\textheight}{9.0 in}
}

\newcommand{\mymarginfix}{
\setlength{\parskip}{0.01cm}
\setlength{\textwidth}{5.0in}
%latex measures from 1 in down and over for some reason
\setlength{\oddsidemargin}{1.0 in}
\setlength{\evensidemargin}{0 in}
\setlength{\topmargin}{-0.8in}
\setlength{\textheight}{6.5 in}
}
\marginfix


%===================================================================%
% Bibliography
%===================================================================%
\let\oldthebibliography=\thebibliography
  \let\endoldthebibliography=\endthebibliography
  \renewenvironment{thebibliography}[1]{%
    \begin{oldthebibliography}{#1}%
      \setlength{\parskip}{.3ex}%
      \setlength{\itemsep}{.3ex}%
  }%
  {%
    \end{oldthebibliography}%
  }


%===================================================================%
% Paragraph, captions, margins
%===================================================================%
\addtolength{\partopsep}{-1mm}
\addtolength{\itemsep}{-4mm}
\addtolength{\abovedisplayskip}{-2mm}
\addtolength{\belowdisplayskip}{-2mm}
\addtolength{\belowcaptionskip}{-1mm}
\addtolength{\abovecaptionskip}{-1mm}
\addtolength{\textfloatsep}{-1mm}
%\input psfig
%\usepackage{psfig}

\setlength{\marginparwidth}{2cm}



%\reversemarginpar

%\long\def\authornote#1{%
%        \marginnote{\it #1}}
%\newcommand{\thomas}[1]{\authornote{TAF: #1}}
%\renewcommand{\thomas}[2][]{}


%===================================================================%
% \fillmeup command
%===================================================================%
%%%
\newcommand{%
\fillmeup}{%
\vspace{.1in}%
\\%
}%


%===================================================================%
% Typesetting MATLAB code
%===================================================================%

\lstset{language=Matlab,%
    %basicstyle=\color{red},
    breaklines=true,%
    morekeywords={matlab2tikz},
    keywordstyle=\color{blue},%
    morekeywords=[2]{1}, keywordstyle=[2]{\color{black}},
    identifierstyle=\color{black},%
    stringstyle=\color{mylilas},
    commentstyle=\color{mygreen},%
    showstringspaces=false,%without this there will be a symbol in the places where there is a space
    numbers=left,%
    numberstyle={\tiny \color{black}},% size of the numbers
    numbersep=9pt, % this defines how far the numbers are from the text
    emph=[1]{for,end,break},emphstyle=[1]\color{red}, %some words to emphasise
    %emph=[2]{word1,word2}, emphstyle=[2]{style},    
}



